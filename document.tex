\documentclass[wi]{zut}

\usepackage{caption}
\usepackage{subcaption}
\usepackage{microtype}
\usepackage{wrapfig}
\usepackage{lipsum}

% Captions setup
\captionsetup[figure]{labelfont=bf, font=small}
\captionsetup[lstlisting]{labelfont=bf, font=small}
\captionsetup[table]{labelfont=bf, font=small}
\captionsetup[subfigure]{labelformat=empty, font=small, justification=centering}
\captionsetup{justification=raggedright,singlelinecheck=false}

% Enumerations setup
\usepackage{enumitem}
\setlist[itemize]{noitemsep, topsep=0pt}
\setlist[enumerate]{noitemsep, topsep=0pt}

% My commands
% https://tex.stackexchange.com/questions/429876/icons-on-side-of-text-below-section-headings/430016#430016
% https://tex.stackexchange.com/questions/459221/how-to-add-an-icon-to-a-paragraphs-margin
% https://tex.stackexchange.com/questions/87324/how-can-i-add-side-icons-in-a-latex-book
\newcommand{\question}{
    \marginpar{\includegraphics[scale=0.025]{graphic/question_mark.png}\hfill}
}

\author{Karol Działowski}
\title{Pytania na obronę pracy magisterskiej}

\makemetadata

\begin{document}

\maketitle
\tableofcontents

\section{Wstęp}

Poniższe opracowania były opracowywane indywidualnie i nie są oficjalnymi ani sprawdzonymi odpowiedziami na pytania. Na wiele pytań nie jestem pewien udzielonej odpowiedzi -- przy nich widnieje znak zapytania jak na przykładzie niżej.

W tym akapicie przedstawiony jest schemat oznaczenia pytań co do których odpowiedzi nie mam pewności. Po prawej stronie będzie wklejona ikona znaku zapytania.
\question

Podczas opracowywania pytań starałem się tłumaczyć możliwie jak najogólniej powiązane zagadnienia. W wielu przypadkach też zamieściłem odnośniki do źródeł z których korzystałem podczas tworzenia z tego dokumentu.

% \section{Przedmioty wspólne}

% \subsection{Zasady cyfryzacji sygnałów. Prawo Kotielnikowa-Shannona. Granica Nyquisa. Aliasing.}

% \subsection{Bezpieczny schemat podpisu cyfrowego. Modele bezpieczeństwa.}

% \subsection{Idea interpolacji funkcji z wykorzystaniem funkcji sklejanych.}

% \subsection{Styl poznawczy (kognitywny) człowieka}

% \subsection{Korzyści wynikające z zastosowania grafowych baz danych do przetwarzania dużych zbiorów danych o strukturach grafowych}

% \subsection{Założenia i obszary zastosowania platformy Apache Spark}

% \subsection{Sposoby sprawdzenia właściwości losowych danego ciągu}

% \subsection{Filtracja cyfrowa: filtry SOI i NOI}

% \subsection{Reprezentacja sygnałów za pomocą szeregów funkcyjnych. Dyskretne transformacje ortogonalne oraz szybkie algorytmy ich wyznaczania.}

% \subsection{Atak na podpis cyfrowy wykorzystujący paradoks dnia urodzin}

% \subsection{Podział metod rozwiązywania równań liniowych metodami numerycznymi}

% \subsection{Algorytm numeryczny niestabilny a algorytm źle uwarunkowany}

% \subsection{Programowanie kodu wielowątkowego wraz z wzajemnym wykluczaniem w C++ 11 Threads}

% \subsection{Cechy środowiska Hadoop}

% \subsection{Rodzaje błędów składające się na całkowity błąd obliczeń numerycznych}

% \subsection{Standard C++ 17 Parallel}

% \subsection{Dokładność rozwiązywania równań różniczkowych metodami numerycznymi}

% \subsection{Wpływ zachowań z obszaru kognitywistyki relacji społecznych na rozwój mediów społecznościowych}

% \subsection{Programowanie przenośnego kodu wielowątkowego na przykładzie PosixThreads}

% \subsection{Zastosowanie okulografii w pięciu wybranych dziedzinach życia}

\section{Inteligencja obliczeniowa}

% \subsection{Charakterystyka języków programowania wykorzystywanych w analizie danych}

% \subsection{Porównanie dwóch dowolnych algorytmów wykrywania obiektów}

% \subsection{Charakterystyka wybranych metod śledzenia obiektów}

% \subsection{Trzy pytania, na które odpowiadają Ukryte Modele Markowa oraz używane do tego celu algorytmy}

% \subsection{Cechy obiektów audio i metody ekstrakcji cech tych obiektów}

% \subsection{Przebieg uczenia ze wzmocnieniem i pozyskiwana w tym procesie wiedza}

% \subsection{Cechy charakterystyczne splotowych sieci neuronowych}

% \subsection{Podstawowe różnice między sygnałem mowy a sygnałem muzycznym w dziedzinie częstotliwości}

% \subsection{Meotdy próbkowania sieci złożonych}

% \subsection{Elementy składowe procesu klasyfikacji sygnałów akustycznych}

% \subsection{Model SVM (procedura uczenia, wariant liniowy i nieliniowy, przekształcenia jądrowe).}

% \subsection{Sieci perceptronowe i metody uczenia perceptronu (warianty algorytmów uczenia, głosowanie, zastosowania)}

% \subsection{Proces tworzenia zbioru uczącego, walidującego i testowego w głębokim uczeniu}

% \subsection{Sposób działania algorytmów uczących AdaBoost i RealBoost}

% \subsection{Omówienie na przykładzie algorytmu Apriori odkrywania asocjacji w zbiorach danych}

% \subsection{Przykład siedzi bayesowskiej (przekonań): struktura sieci, właściwości i jej interpretacja oraz uczenie}

\subsection{Deskryptory cech niskopoziomowych - wybrane algorytmy w odniesieniu do wykorzystywanych cech}

Deskryptory cech odnoszą się do zagadnienia ekstrakcji cech. Deskryptor opisuje daną cechę za pomocą wartości numerycznych. 

\textbf{Atrybuty niższego poziomu abstrakcji} - metadane typu sygnałowego, są wartościowane przez komputer (np. kolor dominujący, histogram krawędzi, aktywność ruchu w obrazie, czy linia melodyczna utworu muzycznego) \cite{Frejlichowski2020}.

Standard MPEG-7 opisuje między innymi deskryptory wizualne. Deskryptory wizualne MPEG-7 opisują na \textbf{niewielu} bitach obrazy, sekwencje obrazów, obszary w obrazie itd. 

Deskryptor powinien być:

\begin{itemize}
    \item efektywny i ekspresyjny (porównywalny z widzeniem u ludzi),
    \item zwarty (w aspekcie pamięci),
    \item o małej złożoności ekstrakcji i zapytań \cite{Frejlichowski2020}.
\end{itemize}

Wyróżniamy następujący podział deskryptorów:

\begin{itemize}
    \item \textbf{Koloru} - Scalable Color, Color Structure, Dominant Color, histogramy,
    \item \textbf{Kształtu} - sygnatura, UNL, UNL-F, mUNL,
    \item \textbf{Odcieni szarości} - Polar-Fourier Greyscale Descriptor, Rzutowanie wartości,
    \item \textbf{Ruchu},
    \item \textbf{Tekstury}.
\end{itemize}

\subsubsection{Deskryptory kształtu}

Główne problemy z deskryptorami kształtu to:

\begin{itemize}
    \item obrót, skalowanie, przesunięcie (przekształcenia afiniczne)
    \item szum,
    \item nieciągłości,
    \item okluzja.
\end{itemize}

Czyli najlepszy deskryptor jest inwariantny względem obrotu, skalowania, przesunięcia oraz jest niewrażliwy na szum i okluzję.

Proste deskryptory kształtu to m.in. pole powierzchni, długość obwodu, kołowość, zwartość, mimośród, powłoka wypukła, object aspect ratio, itd. \cite{Frejlichowski2020_2}

Przykładowym deskryptorem jest UNL-F. Zapewnia on dobrą odporność na szum, przekształcenia afiniczne (transformata Fouriera daje odporność na obrót). Jedyną wadą jest brak odporności na okluzję. W pewnym stopniu rozwiązuje to mUNL, który zmienia podejście do wyznaczania centroidu  \cite{Frejlichowski2020_2}.

W skrócie UNL-F składa się on z następujących kroków:

\begin{enumerate}
    \item Binaryzacja obiektu
    \item Wyznaczenie centroidu
    \item Przekształcenie do współrzędnych polarnych
    \item Wycinek widma po transformacie 2D Fouriera -- tego kroku nie ma w zwykłej metodzie UNL
\end{enumerate}

\begin{figure}[H]
    \centering
    \includegraphics[width=0.5\linewidth]{images/unl.png}
    \vspace{1em}
    \caption{Działanie deskryptora UNL}
    \label{fig:pdgd}
    \source{Wykład 2 - D. Frejlichowski \cite{Frejlichowski2020_2}}
\end{figure}

\subsubsection{Deskryptory koloru}

Przykładowe deskryptory koloru ze standardu MPEG-7 to:

\begin{itemize}
    \item Deskryptor koloru dominującego
    \item Skalowalny deskryptor koloru
    \item Deskryptor GOF i GOP
    \item Deskryptor struktury koloru
    \item Deskryptor widoku koloru (layout)
    \item Temperatura barwowa \cite{Frejlichowski2020_6}
\end{itemize}

Inne deskryptory koloru to:

\begin{itemize}
\item histogram RGB lub HSV - może dawać takie same wyniki dla różnych obrazów
\item IBOX8
\item DHV
\item RGBBOX8
\item RGBI \cite{Frejlichowski2020_6}
\end{itemize}

Przykładowo, deskryptor koloru dominującego polega na wyznaczenie $n$ kolorów dominujących dla obrazu za pomocą algorytmu $k$-means i obliczenie ich udziałów \cite{Frejlichowski2020_6}.

Skalowalny deskryptor koloru bazuje na przestrzeni barw HSV i transformacji Haara zastosowanej na wartościach histogramu koloru \cite{Frejlichowski2020_6}.

Deskryptor rozkładu koloru (Color Layout Descriptor) uchwyca rozkład przestrzenny kolorów w bazie. Polega na transformacji DCT dla poszczególnych składowych w przestrzeni YcbCr~\cite{Frejlichowski2020_6}.

\subsubsection{Deskryptory odcieni szarości}

W pewnym sensie deskryptory odcieni szarości łączą w sobie deskryptory kolorów i kształtu. Przykładowym deskryptorem jest Polar-Fourier Grayscale Descriptor.

Polega on na wstępnym preprocessingu (filtry), wyznaczeniu centroidu obrazu w skali szarości, transformacji obrazu do współrzędnych biegunowych i wycięciu fragmentu widma po transformacie 2D Fouriera \cite{frejlichowski2015application}.

Alternatywny deskryptor polegał na przekształceniu biegunowym i rzutowaniu wartości. Rzutowanie to suma po kolumnach i suma po wierszach. Taki deskryptor daje na wyjściu dwa wektory. Dawał gorsze wyniki niż PFGD \cite{Frejlichowski2020_5}.


% \subsection{Główne miary centralności w sieciach złożonych}

% \subsection{Pakiety języka Python wykorzystywane w analizie danych}

% \subsection{Techniki regularyzacji modeli klasyfikacyjnych i regresyjnych (regularyzacja L1, L2, elastic net; problemy optymalizacyjne; zastosowania)}


\printbibliography[heading=bibintoc]

\appendix

\section{Przykłady do kopiowania}

\begin{table}[H]
\caption{Czas trwania jednej epoki dla rozmiaru obrazu}
\vspace{1em}
\centering
\begin{tabular}{@{}lr@{}}
\toprule
Rozmiar          & Czas {[}s{]} \\ \midrule
$32 \times 32$   & 0.1288       \\
$64 \times 64$   & 0.2000       \\
$128 \times 128$ & 1.046        \\ \bottomrule
\end{tabular}
\end{table}

\code{Przykład kodu}
{Opracowanie własne}{\label{kod:przyklad}}
\begin{lstlisting}[language=Python]
discriminator = make_discriminator_model()
generator = make_generator_model()
\end{lstlisting}

\begin{figure}[H]
    \centering
    \includegraphics[width=0.7\linewidth]{images/sample.png}
    \vspace{1em}
    \caption{Przykładowy obrazek}
    \label{fig:pdgd}
    \source{Opracowanie własne}
\end{figure}


\end{document}
